\chapter{Define the problem}

\section{The problem}
Given data from different sensors of a mobile device, the problem is to classify the activity of the person as one of the following: walking, walking upstairs, walking downstairs, sitting, standing, laying)

\section{History of the dataset}
The experiments have been carried out with a group of 30
volunteers within an age bracket of 19-48 years. Each
person performed six activities (walking, walking upstairs, walking downstairs, sitting, standing, laying) wearing a smartphone on the
waist. Using its embedded accelerometer and gyroscope, we
captured 3-axial linear acceleration and 3-axial angular
velocity at a constant rate of 50Hz. The experiments have
been video-recorded to label the data manually. The
obtained dataset has been randomly partitioned into two
sets, where 70\% of the volunteers was selected for
generating the training data and 30\% the test data. 

The sensor signals (accelerometer and gyroscope) were
pre-processed by applying noise filters and then sampled
in fixed-width sliding windows of 2.56 sec and 50\%
overlap (128 readings/window). The sensor acceleration
signal, which has gravitational and body motion
components, was separated using a Butterworth low-pass
filter into body acceleration and gravity. The
gravitational force is assumed to have only low frequency
components, therefore a filter with 0.3 Hz cutoff
frequency was used. From each window, a vector of features
was obtained by calculating variables from the time and
frequency domain. \cite{uci-link}

\section{Attribute information}
For each record in the dataset it is provided:
\begin{enumerate}
    \item Triaxial acceleration from the accelerometer (total acceleration) and the estimated body acceleration.
    \item Triaxial Angular velocity from the gyroscope.
    \item A 561-feature vector with time and frequency domain variables.
    \item A 561-feature vector with time and frequency domain variables.
    \item Its activity label. 
    \item An identifier of the subject who carried out the experiment.
\end{enumerate}

\section{First look at the dataset}

\chapter{Summarize data}

\section{Descriptive statistics}
\section{Visualize the data}
\section{Handling missing values}
\section{Standardization and normalization}
\section{Feature selection}

\chapter{Prepare data}

\section{Raw view of the dataset}
\section{Normalized view of the dataset}
\section{Standardized view of the dataset}
\section{View after feature selection}

\chapter{Evaluate algorithms}

\section{Experiment design 1}
\section{Experiment design 2}
\section{Experiment design 3}
\section{Experiment design 4}

\chapter{Improve results}

\section{Ensemble algorithms}
\section{Parameter tuning}

\chapter{Present results}
\section{Final model location}
\section{Possible extensions}